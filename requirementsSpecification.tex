\documentclass[10pt]{article}
\author{Gustavo Meireles, José Pombo, Vanessa Lai Leong, Tomás Costa}
\title{Software Requirements Specification}
\usepackage{fullpage, float, hyperref}

\begin{document}
\maketitle

\section{Preamble}
Este documento é destinado aos desenvolvedores e clientes da aplicação 
em questão. Esta é a primeira versão deste documento, estando a ser 
escrita para a disciplina Engenharia de Software. 

Esta aplicação será utilizada para permitir o envio, recebimento e 
avaliação de trabalhos entre alunos e professores. 
Estas três funcionalidades são as principais funções da aplicação, 
sendo necessárias para a sua utilidade básica.

\section{Introduction}
A aplicação permite a um aluno enviar trabalhos a um professor 
em específico. O aluno será avisado quando o trabalho em questão 
é recebido pelo professor a quem o trabalho foi enviado. 
Este será capaz de avaliar o trabalho enviado pelo aluno e 
enviar a avaliação ao aluno em questão. Sempre que um aluno 
quiser enviar um documento a um professor, o aluno irá fazer o 
upload do documento para um servidor central que irá consequentemente 
enviar o mesmo documento para o professor especificado pelo aluno. 
Depois de o professor receber o documento, uma notificação é enviada ao 
aluno a dizer que o documento foi recebido com sucesso. 

Depois de realizar a sua avaliação, o professor pode mandar a mesma 
ao aluno, esta avaliação será enviada como um email a partir do 
servidor principal. Este email irá conter a avaliação do professor e 
informará qual o documento avaliado. Depois do envio da avaliação pelo 
professor e o recebimento da mesma pelo aluno, o documento 
guardado no servidor será apagado. Todo este processo é o ciclo 
principal da aplicação, sendo repetido para todos os envios de trabalhos. 
O processo descrito deve ser facilmente realizado com um 
computador e qualquer ficheiro em formato pdf ou qualquer outro
formato guardado em pleno texto (pdf é o formato 
preferível para evitar qualquer possibilidade de acontecerem 
alterações ao documento durante o seu tempo de vida dentro 
do sistema da aplicação).

\newpage

\tableofcontents{}

\newpage

\newpage

\section{Functional Requirements}
\begin{itemize}
    \item User authentication;
        \begin{itemize}
            \item User registration;
            \item User login;
        \end{itemize}
    \item Submission of work;
    \item Receiving of work;
    \item Storing of work:
        \begin{itemize}
            \item Offline storing of work;
            \item Online storing of work;
        \end{itemize}
    \item File tampering verification;
    \item Evaluation of work;
    \item User notifications;
        \begin{itemize}
            \item Professor notifications;
            \item Student notifications;
        \end{itemize}
\end{itemize}

\newpage

\subsection{User authentication}
O sistema deve fornecer autenticação de utilizadores para que os 
professores e alunos possam aceder às suas contas de forma segura. 
Os utilizadores criados e existentes serão também utilizados para 
o funcionamento da aplicação, existindo permissões e proibições diferentes
entre uma conta de aluno e uma conta de professor.

\subsubsection{User registration}
A plataforma não terá qualquer conta inicialmente. Os alunos e professores terão
de criar as suas contas para poderem interagir com qualquer parte da aplicação.
Consequentemente, é necessário ser possível realizar o registo de utilizadores.

O registo de utilizadores necessita do fornecimento de vários dados distintos:
\begin{itemize}
    \item posição:
        \begin{itemize}
            \item professor;
                \begin{itemize}
                    \item turmas
                    \item escolas
                \end{itemize}
            \item estudante;
                \begin{itemize}
                    \item escola
                    \item turma
                        \begin{itemize}
                            \item disciplinas
                        \end{itemize}
                \end{itemize}
        \end{itemize}
    \item email
    \item nome de usuário
    \item nome legal
    \item password
\end{itemize}

Toda esta informação será utilizada pela aplicação para criar as relações
entre contas necessárias para organizar os professores and alunos em turmas
diferentes. Esta organização evita o contacto entre alunos a professores acidental
out indesejado (ex: um aluno enviar um trabalho a um professor com quem não tem aulas).

\newpage

\subsubsection{User login}
Depois de estar registado, um utilizador pode realizar o login a qualquer momento que deseja.
Para realizar o login o utilizador necessita de utilizar dois bocados de informação:
\begin{itemize}
    \item username
    \item password
\end{itemize}
Esta informação será enviada ao servidor para ser validada contra todas as contas encontradas
dentro do sistema. A informação destas contas não é guardada em pleno texto, utilizando
o método de Hash and Salt garantindo segurança para os utilizadores.

\begin{table}[H]
    \centering
    \begin{tabular}{|l|p{0.6\linewidth}|}
        \hline
        \textbf{Function} & Esta função garante que apenas utilizadores autorizados podem aceder ao sistema, verificando a sua 
        identidade através de nome de utilizador e palavra-passe. \\
        \hline
        \textbf{Inputs} & Nome de Utilizador: Fornecido pelo utilizador. \\
                        & Palavra-passe: Fornecida pelo utilizador. \\
                        \hline
        \textbf{Source} & Utilizador a tentar iniciar sessão. \\
        \hline
        \textbf{Outputs} & Autenticação bem-sucedida ou mensagem de erro. \\
        \hline
        \textbf{Destination} & Painel de controlo do utilizador ou página de erro. \\
        \hline
        \textbf{Action} & O sistema verifica o nome de utilizador e palavra-passe introduzidos em relação às credenciais de 
        utilizador armazenadas. \\
                        & Se as credenciais forem válidas, o sistema concede acesso à conta do utilizador. \\
                        & Se as credenciais forem inválidas, o sistema apresenta uma mensagem de erro. \\
                        \hline
        \textbf{Requirements} & Os dados de registo do utilizador (nome de utilizador e palavra-passe) devem ser armazenados de 
        forma segura (Hash and Salt) no sistema. \\
        \hline
        \textbf{Pre-Condition} & O utilizador tem uma conta registada no sistema. \\
                               & O utilizador pretende iniciar sessão. \\
                               \hline
        \textbf{Post-Condition} & Se a autenticação for bem-sucedida, o utilizador obtém acesso à sua conta e pode 
        realizar ações autorizadas. \\
                                & Se a autenticação falhar, o utilizador é notificado do erro e não obtém acesso ao sistema. \\
                                \hline
    \end{tabular}
\end{table}

\newpage

\subsection{Submission of work}
Os professores devem ser capazes de receber submissões eletrónicas de 
trabalhos dos estudantes. Os estudantes devem ter a capacidade de 
submeter trabalhos através do sistema. Os trabalhos devem suportar 
formatos de documentos comuns (por exemplo, PDF, Word).

Para garantir que os trabalhos não são modificados, todos os documentos enviados pela
plataforma, serão verificados pela utilização de Hash Codes.
Casa ficheiro tem o seu próprio hash, sendo este modificado se qualquer tipo de modificação
acontecer ao ficheiro.

\begin{table}[H]
    \centering
    \begin{tabular}{|l|p{0.6\linewidth}|}
        \hline
        \textbf{Function} & Esta função permite que tanto os professores como os estudantes submetam trabalhos eletronicamente 
        através do sistema. Garante que os trabalhos submetidos suportam formatos de documento comuns, como PDF e Word. \\
        \hline
        \textbf{Inputs} & Ficheiro do trabalho submetido (PDF, Word ou formato suportado): 
        Fornecido pelo utilizador (professor ou estudante). \\
        \hline
        \textbf{Source} & Utilizador (professor ou estudante) a iniciar a submissão do trabalho. \\
        \hline
        \textbf{Outputs} & Confirmação de submissão de trabalho bem-sucedida ou uma mensagem de erro. \\
        \hline
        \textbf{Destination} & Mensagem de confirmação ou mensagem de erro dentro do sistema. \\
        \hline
        \textbf{Action} & O sistema aceita e processa o ficheiro do trabalho submetido. \\
                        & Verifica se o formato do ficheiro é suportado (por exemplo, PDF, Word). \\
                        & Se o formato for suportado, o sistema confirma a submissão bem-sucedida. \\
                        & Se o formato não for suportado ou houver algum problema com a submissão, o sistema apresenta uma mensagem de erro. \\
                        \hline
        \textbf{Requirements} & O sistema deve ter a capacidade de receber e processar submissões eletrónicas de documentos. \\
                              & Os formatos de documento suportados (por exemplo, PDF, Word) devem ser definidos e implementados. \\
                              \hline
        \textbf{Pre-Condition} & O utilizador (professor ou estudante) tem uma conta ativa e está autenticado no sistema. \\
                               & O utilizador pretende submeter um trabalho. \\
                               \hline
        \textbf{Post-Condition} & Se a submissão for bem-sucedida e o formato do documento for suportado, o sistema confirma a 
        submissão bem-sucedida. \\
                                & Se houver problemas com a submissão (por exemplo, formato não suportado ou falha no carregamento), o sistema 
                                apresenta uma mensagem de erro, e a submissão não é processada. \\
                                \hline
    \end{tabular}
\end{table}

\newpage

\subsection{Storing of work}
Todos os ficheiros enviados pela aplicação serão guardados apenas
durante a sua utilidade para avaliação e entrega da mesma.
Cada ficheiro terá um código hash único e será encriptado para assegurar
segurança dos trabalhos.

Cada trabalho será guardado assim que o aluno envie o mesmo para um professor.
Antes do trabalho entrar no sistema, será criado um código hash com os seu estado
atual e será encriptado, sendo apenas possível abrir o ficheiro depois de ser feito
o login como o professor para que o documento foi enviado.

Depois de ser realizada a avaliação, a mesma será guardada separadamente do trabalho avaliado.
Tendo apenas capacidade de ser visualizada se for aberta em conjunto com o trabalho correspondente
pelo aluno correspondente.

Depois de todos este paços serem executados, o documento será eliminado, deixando apenas a
avaliação para ser utilizada como os professores desejarem.

\subsubsection{Details}
Os trabalhos irão ser guardados numa  base de dados central. Todos os binários enviados serão guardados
de acordo com os pontos feitos anteriormente. Sabendo que a maior parte dos ficheiros apenas são enviados
e abertos uma vez, não é necessário uma grande solução de caching para esta aplicação.

Uma base de dados relacional consegue preencher todos os casos de utilização possíveis da aplicação.
Consequentemente, a nossa solução de \textit{storage} será extremamente simples, assim evitando várias
complicações causadas por sistemas em que temos de partilhar informação entre várias bases de dados diferentes.

\newpage

\subsection{Evaluation of Work}
Os professores devem ter a capacidade de avaliar e classificar trabalhos. 
Os critérios de avaliação (por exemplo, rubricas) devem ser personalizáveis. 
Os professores devem poder fornecer comentários e atribuir pontuações a cada trabalho. 
O sistema deve calcular e armazenar as pontuações globais dos trabalhos.

\begin{table}[H]
    \centering
    \begin{tabular}{|l|p{0.6\linewidth}|}
        \hline
        \textbf{Function} & Esta função permite aos professores avaliar e classificar trabalhos submetidos pelos estudantes. 
        Permite critérios de avaliação personalizáveis (por exemplo, rubricas) e oferece aos professores a capacidade de 
        fornecer comentários e pontuações para cada trabalho. O sistema também calcula e armazena as pontuações globais 
        dos trabalhos. \\
        \hline
        \textbf{Inputs} & Trabalho submetido (pelos estudantes). \\
                        & Critérios de avaliação (personalizáveis, por exemplo, rubricas) fornecidos pelo professor. \\
                        & Comentários e pontuações fornecidos pelo professor. \\
                        \hline
        \textbf{Source} & Professores a aceder aos trabalhos submetidos pelos estudantes. \\
        \hline
        \textbf{Outputs} & Pontuações de avaliação e feedback para cada trabalho. \\
                         & Pontuações globais para cada trabalho. \\
                         \hline
        \textbf{Destination} & Registo do estudante no sistema (para feedback individual). \\
                             & Armazenamento/base de dados do sistema (para pontuações globais). \\
                             \hline
        \textbf{Action} & Os professores reveem os trabalhos submetidos. \\
                        & Utilizam critérios de avaliação personalizáveis (por exemplo, rubricas) para avaliar os trabalhos. \\
                        & Os professores fornecem comentários e atribuem pontuações a cada trabalho. \\
                        & O sistema calcula as pontuações globais com base nos critérios fornecidos. \\
                        & As pontuações e feedback são armazenados para cada trabalho. \\
                        \hline
        \textbf{Requirements} & Critérios de avaliação personalizáveis (por exemplo, rubricas) devem ser definidos e 
        disponíveis no sistema. \\
                              & O sistema deve suportar o armazenamento de dados de avaliação. \\
                              \hline
        \textbf{Pre-Conditions} & Os professores têm acesso aos trabalhos submetidos. \\
                                & Os critérios de avaliação estão estabelecidos. \\
                                \hline
        \textbf{Post-Conditions} & Os trabalhos individuais têm associadas pontuações de avaliação e feedback. \\
                                 & As pontuações globais para cada trabalho são calculadas e armazenadas. \\
                                 & Os estudantes podem aceder ao seu feedback individual. \\
                                 & Os professores podem gerir e rever as avaliações para cada trabalho. \\
                                 \hline
    \end{tabular}
\end{table}

Para este tipo de avaliação ser possível, será apresentada uma interface gráfica
ao professor.
Esta interface será mapeada de acordo com o documento que está a ser avaliado
linha por linha. Ao ser inserido um comentário pelo professor, uma cotação ou
uma correção, será guardada a posição da mesma, sendo possível assim, reconstruir
a correção do professor quando esta for ser vista pelo aluno. Isto evita a necessidade
de modificar os documentos enviados pelos alunos em si.

\newpage

\subsection{Notifications}
O sistema deve notificar os estudantes automaticamente quando os seus trabalhos forem avaliados. 
As notificações devem incluir os resultados da avaliação e os comentários. 

As notificações serão simples notificações de IOS/Android/Windows/MacOS/Linux, utilizando
os sistemas de notificações oferecidos por estas plataformas.

\begin{table}[H]
    \centering
    \begin{tabular}{|l|p{0.6\linewidth}|}
        \hline
        \textbf{Function} & Esta função envolve a notificação automática dos estudantes quando os seus trabalhos forem avaliados. 
        As notificações devem incluir os resultados da avaliação e o feedback. \\
        \hline
        \textbf{Inputs} & Resultados da avaliação e feedback gerados durante a avaliação do trabalho. \\
        \hline
        \textbf{Source} & O sistema, especificamente o processo de avaliação de trabalhos. \\
        \hline
        \textbf{Outputs} & Mensagens de notificação enviadas aos estudantes correspondentes. \\
        \hline
        \textbf{Destination} & Canais de comunicação dos estudantes (por exemplo, email, notificações no sistema). \\
        \hline
        \textbf{Action} & Após a conclusão da avaliação dos trabalhos, o sistema gera mensagens de notificação. \\
                        & Estas mensagens incluem os resultados da avaliação (pontuações) e o feedback. \\
                        & O sistema envia estas notificações aos respetivos estudantes. \\
                        \hline
        \textbf{Requirements} & Um mecanismo para a geração de notificações automáticas. \\
                              & Acesso às informações de contacto dos estudantes (por exemplo, endereços de email). \\
                              \hline
        \textbf{Pre-Conditions} & As avaliações dos trabalhos foram concluídas pelos professores. \\
                                & As informações de contacto dos estudantes estão disponíveis e atualizadas. \\
                                \hline
        \textbf{Post-Conditions} & Os estudantes recebem notificações automáticas contendo os resultados da avaliação 
        dos seus trabalhos e o feedback. \\
                                 & Os estudantes podem aceder aos seus resultados de avaliação e ao feedback através dos seus canais de comunicação 
                                 escolhidos (por exemplo, email, notificações no sistema). \\
                                 \hline
    \end{tabular}
\end{table}

\subsubsection{Notifications for students}
Cada notificação para estudantes será personalizada de acordo com o trabalho
que representa, disciplina e professor a quem foi enviado.
Uma notificação poderá conter mensagens como os seguintes exemplos:

\begin{itemize}
    \item Trabalho enviado a professor:
        \begin{itemize}
            \item O trabalho \textit{x} foi enviado ao professor \textit{y} | 19:04.
        \end{itemize}
    \item Trabalho recebido por professor:
        \begin{itemize}
            \item O trabalho \textit{x} foi recebido pelo professor \textit{y} | 19:06.
        \end{itemize}
    \item Avaliação recebida por aluno:
        \begin{itemize}
            \item Recebeu uma avaliação nova do trabalho \textit{x} | 19:07
        \end{itemize}
\end{itemize}

\subsubsection{Notifications for professors}
Tal como para os estudantes, a informação apresentada nas notificações
dos professores irá variar de acordo com os trabalho recebido e condições
como o tempo a que as mesmas foram enviadas:
\begin{itemize}
    \item Novo trabalho para avaliar recebido:
        \begin{itemize}
            \item Trabalho \textbf{novo} recebido de aluno \textit{x} da turma \textit{y} da disciplina \textit{z}. | 19:10
        \end{itemize}
    \item Avaliação recebida:
        \begin{itemize}
            \item A \textbf{avaliação} do trabalho \textit{x} foi recebida pelo aluno \textit{y} da turma \textit{z}
                da disciplina \textit{h} | 19:16
        \end{itemize}
\end{itemize}

Para além das notificações explicadas até agora, a aplicação não deve enviar qualquer outro tipo de notificação
aos seus utilizadores. Isto é para evitar o envio excessivo de notificações que acontece na industria.

\newpage

\section{Non-Functional Requirements}

\subsection{Performance}
O sistema deve ser capaz de lidar com um grande número de submissões de trabalhos e avaliações simultaneamente. 
Os tempos de resposta para aceder aos trabalhos e avaliações devem ser rápidos.

Os tempos de resposta serão alcançados pela implementação de um sistema de caching local.
Este sistema irá fazer o download de todos os documentos recebidos ao ligar a aplicação ao invés
de fazer o download apenas quando os documentos estiverem a ser abertos. Este download deve ser
feito assincronamente para evitar o bloqueio da funcionalidade do resto da aplicação. Permitindo
a interação com os resto das funcionalidades da aplicação em tempo real durante o download de outros
ficheiros que não  se encontram a ser utilizados.

\subsection{Security}
Os dados dos utilizadores, trabalhos e avaliações devem ser armazenados de forma segura. 
A autenticação de utilizadores deve utilizar métodos de encriptação seguros. 
Apenas utilizadores autorizados (professores e estudantes) devem ter acesso a dados introduzidos no sistema por outros 
utilizadores.

Os métodos utilizados para assegurar a segurança dos utilizadores e dos documentos dentro da aplicação já forem explicados
anteriormente neste documento.

Para assegurar que os documentos não são modificados depois de entrarem dentro do sistema da aplicação, irá ser
utilizado um sistema de hashing que identificará o estado do documento quando este entra na aplicação pela primeira
vez. Qualquer mudança ao documento (ex, nome) será refletida numa mudança do seu código hash. Isto permite que a
aplicação consiga assegurar a integridade de todos os documentos aos seus utilizadores, deixando estes saber que
não é necessário desconfiar os documentos enviados pela aplicação.

Em termos de segurança de utilizadores e das suas contas, toda a informação de todos os utilizadores será
guardada e protegida utilizando um método do tipo \textit{Hash and Salting}, isto evita que uma falha na
segurança da base da dados não signifique o comprometimento dos dados dos utilizadores.

\subsection{Usability}
A interface do utilizador deve ser intuitiva e de fácil utilização.
Professores e estudantes devem conseguir navegar facilmente e executar tarefas dentro do sistema.

Tendo em conta que é impossível criar uma interface e \textit{user experience} que seja adequada a todos os
utilizadores, a aplicação terá duas maneiras de interação:
\begin{itemize}
    \item Interação gráfica.
    \item Interação por linha de comandos.
\end{itemize}

\subsubsection{Interação gráfica}
A interação gráfica será a maneira principal de interagir com a aplicação e as suas funcionalidades. O design e princípios
utilizados para desenvolver estas interfaces devem seguir todos os princípios de design e regras impostos pelos sistemas
em que a aplicação está a ser desenvolvida, seja isto um ambiente mobile ou um ambiente desktop.

\subsubsection{Interação por linha de comandos}
Para nos assegurar-mos da portabilidade da aplicação a garantir a sua utilidade em sistemas menos suportados e no futuro,
será criada uma interface da aplicação que será utilizada pela linha de comandos.
Esta versão apenas deve utilizar tecnologias que se encontrem em sistemas que são \textit{POSIX compliant} para
garantir a suar portabilidade e funcionalidade em vários sistemas que seguem estes princípios.

Esta versão da aplicação também será utilizada para o desenvolvimento de outras interfaces para o programa. Estas
interfaces não seram desenvolvidas por nós, mas sim pelas empresas que se encontram a utilizar a aplicação e
necessitem de outras maneiras de utilizar a aplicação que não forem desenvolvidas por nós.

\subsection{Scalability}
O sistema deve ser projetado para crescer à medida que o número de utilizadores e submissões aumenta ao longo do tempo.
A utilização de técnologias provadas em marcado será uma necessidade durante a construção do sistema.

\subsection{Stability}
O sistema deve estar disponível e ser fiável, com um tempo de inatividade mínimo.

\paragraph{Cópia de Segurança e Recuperação de Dados:}
Deve haver cópias de segurança regulares de dados (trabalhos, avaliações, contas de utilizador).
Deve existir um plano para a recuperação de dados em caso de falhas no sistema.

\subsection{Notification Delivery}
As notificações aos estudantes/professores devem ser simples e rápidas. 
As preferências de notificação (por exemplo, email, notificações dentro do sistema) devem ser configuráveis pelos utilizadores.

\end{document}
